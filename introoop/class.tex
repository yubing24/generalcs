\documentclass[../main.tex]{subfiles}
\begin{document}
\section{Class}
So far, all the programs that you have written are in \emph{imperative pardigm}:
you are giving specific instructions to computer by writing down statements
and declearing methods. The program will run sequentially in the way that you
organize it. However, sometimes programming like this can be very tedious. For
example, if you implement a sorting algorithm that takes an integer array as
input and returns a sorted array, and your friend wants to use your code on a
list of type double, she has to copy everything that you have written before
except changing the type of the paramenter and return type. If this is not tedious,
what if a group of people want to use your sorting algorithm to sort out a variety
kinds of items: floating-point numbers, strings, toys, cards, ... As you can tell,
if there is a method that can sort everything, everyone can save a lot of time
from repeting writing the same code over and over again!

Object-oriented programming is the solution to this kind of scenario. Object-oriented
programming emphasizes on the reusability of code so that lagacy code can be reused
today. The key to understand object-oriented programming is to understand abstraction.

\subsection{Abstraction}
We are living in a very abstract world today. We have all kinds of electronics
around us: electronic watch, microwave oven, computer, television, and game consoles,
but you perhaps have never (or rarely) asked yourself: how do they work internally?
When you get your new game console, you probably don't need to read the instruction
manual to hook up wires with your television, insert your game disk into the console,
and play your game with your joystick. When you use electronics, you don't care
what is inside of them. All you need to know is that a game console needs power,
has a DVD player, and can be connected to your television through a cable. You also
know that your joysticks have buttons and control sticks so that you can control
your game as indicated on the joystick. In a word, you don't really, and mostly
importantL: you don't want to care about what is inside of them at all. All you
need to know that they are supposed to work in the way like their names indicate.

This is abstraction: you don't have to understand how to wire a joystick to use
it, and you don't need to know how to build a car to drive it. Similarly, you
don't need to know how the function ``Console.WriteLine()'' implemented to use it.

\end{document}
