\documentclass[../main.tex]{subfiles}
\section{Chapter Exercise}
\begin{enumerate}
    \item \emph{Dog Class} If you want to create a class of dogs, which of these
    following attributes should be members, and which should be methods? Why or why not?
    \begin{enumerate}
        \item Hair color
        \item Bark
        \item Breathe
        \item Weight
        \item Eat
        \item Run
        \item Height
        \item Sleep
    \end{enumerate}

    \item \emph{Public and Private} Which of the attributes above should be public?
    Which should be private? Why or why not?

    \item \emph{Bank Account Class} Consider a bank account as a class of objects.
    A typical bank account has checking and savings accounts, and the owner of
    the account can check the balance of each of them, and make deposit or
    withdraw money from both account. Write a class that can be instantiated.

    \item \emph{Candidate Class} Declare a class \texttt{Candidate}, which has attributes including
    \texttt{name}, \texttt{age}, and \texttt{party}. Then declare methods that
    can get the information of these attributes: \texttt{GetName()}, \texttt{GetAge()},
    and \texttt{GetParty()}. Before you get started, think about the data type of
    each attibute, and how you would design and use this class. Once you complete,
    instantiate two candidates: one has name \texttt{"George W. Bush"}, age \texttt{58}, and
    party \texttt{"Republican"}, while the other one has name \texttt{"John Kerry"},
    age \texttt{61}, and party \texttt{"Democrat"}. Use your getter methods to
    check your implementation.
\end{enumerate}
