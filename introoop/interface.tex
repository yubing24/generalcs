\documentclass[../main.tex]{subfiles}
\begin{document}
\section{Interface}
In many cases, instances of different classes have similar behaviour, and this
is quite common in programming and real world. In real world, both human beings
and parrots can ``talk'', but the way they talk are quite different: human
beings actually commit meaningful talks while a parrot simply imitate what we
teach him. However, since they both can ``talk'' in some way, it would be easier
to treat them as a `Talkable' type.

In programming, one way we can do is to define a `Talkable' class, and if we
have `Human' and `Parrot' classes, we can make both of them inherit methods and
members of the `Talkable' classes. But this can be very difficult to be clear to
manage. For example: \emph{add example here!}

To solve the problem of inheriting from multiple classes, we invented the idea
of \emph{interface}. An interface is a collection of method signatures that
specify what methods must be implemented if this interface is going to be used
by a class. Once a class implements all the required methods of this interface,
instances of this class can also be considered as `instances' of this interface,
and thus we achieve the goal of multiple inheritance.

\framebox[1.2\width]{Programming Languages that Support Mutiple Inheritance}

\subsection{Declaration of Interfaces}
To declare interface in \csharp:

\begin{minted}{csharp}
    public interface <NameOfInterface>
    {
        // put method signatures here. Do not implement any of them!
        // Example:
        // public void MethodA ();
        // public int MethodB ();
    }
\end{minted}

For our example mentioned at the beginning of this section, a \texttt{Talkable}
interface can be declared like this:
\begin{minted}{csharp}
    public interface Talkable
    {
        public void Speak (String content);
    }
\end{minted}

\subsection{Using Interface}
As you can tell, this interface only contains a method signature:
\texttt{public void Speak (String content)}, but not define it at all! Yes, the
implementation is left to individual class that is going to use this interface.
For the example that we mentioned at the beginning of this section, we can make
our two classes: \texttt{Human} and \texttt{Parrot} implement this interface:

For \texttt{Human} class:

\begin{minted}{csharp}
    using System;

    public class Human : Talkable
    {
        private String name;

        public Human (String name)
        {
            this.name = name;
        }

        public void Speak (String content)
        {
            Console.WriteLine (this.name + ": " + content);
        }
}
\end{minted}

For \texttt{Parrot} class, we would implement it slightly different so that it
reflects the
property of a parrot:

\begin{minted}{csharp}
    using System;

    public class Parrot : Talkable
    {
        private String name;

        public Parrot (String name)
        {
            this.name = name;
        }

        public void Speak (String content)
        {
            Console.WriteLine ("Hi, I am " + this.name + "!");
            Console.WriteLine ("Hi, I am " + this.name + "!");
            Console.WriteLine ("Hi, I am " + this.name + "!");
        }
    }
\end{minted}

Then, in the main program, we can instantiate a human and a parrot, and make
them talk:

\begin{minted}{csharp}
    using System;
    public class Program
    {
        public static void Main (String [] args)
        {
            Talkable peter = new Human ("Peter");
            Talkable bob = new Parrot ("Bob");

            peter.Speak ("Today's lecture is about the life of Caesar...");
            bob.Speak ("I want to say something different!");
        }
    }
\end{minted}

\subsection{Benefits and Challenges of Using Interface}
The most obvious benefit of using interface instead of multi-inheritance is that
interfaces can be easier to manage than multi-inheritance could. However, there
are some issues with using interfaces as well. For example, what if we have two
interfaces like these and how should we implement them?

\begin{minted}{csharp}
    public interface MyInterfaceA
    {
        public void Foo ();
    }
\end{minted}

and

\begin{minted}{csharp}
    public interface MyInterfaceB
    {
        public void Foo ();
    }
\end{minted}

If we have a class that wants to implement both interfaces, how do we know which
\texttt{Foo ()} method that we are implementing? Even if we can tell that, when
we instantiate the class and invoke a \texttt{Foo ()} method of that instance,
which interface does that \texttt{Foo ()} belongs to?
\end{document}
