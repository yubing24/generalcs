\documentclass[../main.tex]{subfiles}
\begin{document}
\section{Introduction}
Object-Oriented programming is a programming paradigm that treats everything in
a program as an ``object'', thus the entire program is just a composition of
interactions between many objects. Object-oriented paradigm has a long history
in programming, and it remains the most important paradigm in industry today.
Many areas of computing, such as graphics and gaming, benefit a lot from this
paradigm. This chapter introduces some basic concepts of object-oriented programming,
including \emph{class}, \emph{instance}, \emph{inheritance}, \emph{interface},
and \emph{polymorphism}.
\end{document}
