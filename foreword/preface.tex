\documentclass[../main.tex]{subfiles}
\begin{document}
    \par Compared with other mature science subjects, such as physics and chemistry,
    computer science is still in its childhood stage. Young sciences always brings
    issues in teaching since not many people have taught these subjects before.
    Since the first bachelor's degree in computer sciece was awarded, computer
    science have witnessed huge leap in human technology. However, while the
    technology is rapidly changing, the teaching of computer science, espeically
    introductory computer science, is still in a mist. What contents are taught,
    and how they are taught, vary by schools dramatically. A student who took CS
    101 in a univeristy in Midwest may not even understand a word of the first few
    lectures of CS 102 in univeristies of West Coast. The outcome is, all students
    will receive computer science degree in the end, but their knowledge are not
    necessarily equivalent, even fragmented, since school A might put effort on
    cultivating student's programming skills thoroughly, while school B emphasizes
    more on the mathematical part of computer science.

    \par That is the main motivation of writing this book: to offer a relatively
    comprehensive and flexible textbook for CS 101 courses. Generally speaking,
    most CS 101 courses in the United States cover these topics: imperative
    programming and object-oriented programming. Other than these, schools
    usually customize their own curriculums. Some of them may introduces more about
    object-oriented programming and software engineering, while others may directly
    jump into some elementary data structure and algorithms. All these curriculum
    designs have their rationale behind. Therefore, it can be concluded that a
    comprehensive but elementary textbook for CS 101 should be able to satisify most
    schools' needs, and all students have the same opportunity to learn the most of it.

    \par This book is designed with the aim to solve issues mentioned ahead. This book
    starts with basics concepts of type system and elementary discrete mathematics.
    After a brief introduction to the C\# programming language, this book jumps into
    the basics of imperative programming. Several elementary data structures,
    including array, list, queue, and stack are introduced before methods
    (functions) are covered. The last part of this book introduces the
    object-oriented programming paradigm and some fundamental I/O. The structure of
    this book was fundamentally following the CS 302 curriculum of the University of
    Wisconsin-Madison, where I took my introductory computer science.

    \par Besides of the curriculum design, this book is published online through GitBook.
    This idea is from the inspiration of a pair of professors of Madison as well:
    Professor Andrea and Remzi Arpaci-Dusseau, who wrote a great undergraduate
    operating system textbook: Operating Systems: Three Easy Pieces
    and released it online as free book for everyone. Though I have never took a
    course from either of them, I strongly agree with their philosophy about free
    textbook, especially when the computer science is changing rapidly and textbook
    has became a burden for most students.

    \par Finally, I would like to thank to people who have supported me writing this book
    and making this book possible. This book will be constantly updated as
    technology outdates rapidly. I wish every reader could enjoy the fun that
    computer science bring you as I did.

    \emph{To Instructors}
    This book is desinged with flexibility. A one-semester course should be able
    to finish the basic idea of file I/O. The last chapter, software testing, is
    optional for teaching and should be left to students to figure out on their
    own time.

    Besides of the two major programing paradigms, this book also touches a little
    bit of many other related fields of computer science, especially data structure
    and algorithm. However, instrutors should feel free to skip them if they are
    not required by the course or under time constraint.

    \emph{To Students}



    \par Yubing Hou
    \par February 1st, 2015

\end{document}
