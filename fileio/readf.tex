\documentclass[../main.tex]{subfiles}
\begin{document}
\subsection{Read a File}
Since file is nothing but a sequence of character or bits, thus it make sense to
read them sequentially: from the first character/bit all the way until the last
character, which is EOF (End of File) in Unix system.

Before we can read a file, we must specify which file we want to read and where
it is located. You usually need a to specify a path to that file, and the
operating system will guide your program and show where that file is.

In C$\sharp$, to read a file, you need to specify the path to that file.

\subsection{Absolute Path and Relative Path}
There are two ways to specify the path of a file in a system: absolute path and
relative path. They are named ``absolute'' and ``relative'' is because the location
of a particular file can be different based on the ``reference frame'' that we
specify. An absolute path is the path that is from the root directory to
that file, and a relative path is the path that is from current working directory
to that file. For example, in a typical Unix operating system, our target file is
under Documents/ directory and is named as `myfile.txt'. The absolute file path
would be like this: `/home/user/Documents/myfile.txt'.

Suppose our current working directory is under the Desktop/ directory, and we want
to use the file myfile.txt using relative path, we would use: `../Documents/myfile.txt'.

The major difference between an absolute file path and a relative one is that:
the absolute file path is never change, while the relative path changes according
to the path of current working directory.

\subsection{File Reading Algorithm}
A typical file reading algorithm is like this:
    \begin{enumerate}
        \item Get the path to target file.
        \item Check if file exists.
            \begin {itemize}
                \item If yes, proceed to next step.
                \item If no, create that file or handle exception.
            \end {itemize}
        \item Check if program has permission to access that file.
            \begin {itemize}
                \item If yes, proceed to next step.
                \item If no, ask for permission or handle exception.
            \end{itemize}
        \item Use the given path to create a file object (Java or C$\sharp$) or
        create a file descriptor (C/C++)
        \item Perform whatever actions that are required by the program on that file.
        (NOTE: exceptions may happen here as well!)
        \item Close the file.
    \end{enumerate}

    Let's write a simple C$\sharp$ program that opens and print out the content of a
    text file. Suppose you have a file named students.txt that contains the name of
    students in a class. We can simply write a program like this:

    \begin{minted}{csharp}
        using System;
        using System.IO;

        public class ReadFile
        {
            public static void Main(string [] args)
            {
                    string path = "students.txt";
                    StreamReader sr = new StreamReader(path);
                    string line = sr.ReadLine();
                    while (line != null) {
                        Console.WriteLine(line);
                        line = sr.ReadLine();
                    }
                    sr.Close();
            }
        }
    \end{minted}

    You may noticed quite a few flaws of the program above:
    \begin{enumerate}
        \item It can only read the file whose name is ``students.txt'', and this
        file must be in the same directory as the executable program. If either
        of those conditions fail, the program will not run at all.
        \item It does not check if file exists or not. If file does not exist, this
        program is very likely to crash.
    \end{enumerate}

    In order to handle the issues listed above, we need to improve the robustness
    of our program. An intuitive solution would be like this:
    
\end{document}
