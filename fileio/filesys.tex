\documentclass[../main.tex]{subfiles}
\begin{document}
    \section{File System}
    File system is a mechanism that is provided by the operating system to store
    data persistently. Paper, songs, and movies that are stored on your computer
    are files. These kind of files can be read or written by their corresponding
    programs, such as a word processor or music editor. Programs are also made
    of many files, and they are stored on your computer for you to use.

    Programs are special kind of file: they are not like your paper or music to
    be edited. Instead they are supposed to be \emph{executed} by the computer.
    When a program is executed, it needs a \emph{directory} to run under. For
    example, if you double click your music file, your defalut music player will
    start, and it is very likely that it only recognize this file in this directory,
    and may only identify other music files that you put in this directory. If you
    have other music files in other directory, it is not likely for that music
    player to find that file unless you add that music file to the library of
    your music player, in which case your music player will memorize the path
    to that file and the next time you start it up it will look up that file
    from the path you gave.

\end{document}
