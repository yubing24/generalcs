\documentclass[../main.tex]{subfiles}
\begin{document}
\section{File}
\subsection{Definition of File}
So far, all the data that our programs use reside in the memory of a computer,
which means that once program exits, all the data that you had in your program
will be lost. This is not desirable if you want to save the data for later use.
File, as a mechanism of non-volatile storage, solves this problem. File can be
considered as a sequence of bits or characters stored in a non-volatile media,
such as magnetic disk or tape, optical disks (such as CD/DVD), and solid-state
drives. Data stored in these kind of media are not volatile and data can be
loaded to memory if a program decides to use them. However, since these storage
media is usually slower than the speed of memory, reading and writing to these
media can be very slow.

The classical definition of a file is a stream of character. Such file is
usually represented as text file in many operating system. You can imagine that
a file is a long list of characters. For example, if you open your text editor
on your computer, write ``Hello world!'' and save it as hello.txt, then it
probably would look like this: (put an image here).

Visiting a file is a complex process. When a program needs to visit a file, the
operating system will first check and see if that program has enough privilege to
access that file. Some files can be read and write only, while other files, such
as system files, are only for reading by normal users and are editable for
priviledged users (such as system administrator).

Once the operating system confirms that program has enough privilege to access
that file, the operating system returns the control of computer to that program.
Program will open that file and do whatever it is allowed to do on that file.
Program should, but does not have to, close that file once it is done with using
that file. If the program forgets to close that file, the operating system will
close the file for the program once it exits.

The algorithm for a program to access a file should be like this:
\begin{enumerate}
    \item Obtain the file path and name
    \item Check if that file exists
    \item Try to open that file and read it..
    \item Do whatever it is allowed or commanded to do on that file.
    \item Close the file once work is done.
\end{enumerate}

However, if any other issue happens during the process above, the program usually
crashes and exits. This is mostly because some exceptions happen while trying to
access that file and that exception is not handled anywhere in that program. In
the later section of this chapter, we will discuss how to handle exceptions in
program.

\end{document}
