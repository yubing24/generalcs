\documentclass[../main.tex]{subfiles}
\begin{document}
\section{If Statement}
Sometime we only want to execute parts of our program under certain conditions.
For example, we want to read a number from user input, and we want to print it
out if it is an even number and do nothing if the input is an odd number. Therefore,
we would want the print statement to execute only when the input number is even.
Therefore, we would need some mechanism in our program such that the program can
test if certain condition is met. This mechanism is the \emph{if-statement}.

In C$\sharp$, an if-statement code block looks like this:
\begin{minted}{csharp}
    if (<condition>)
    {
        <statement #1>;
        <statement #2>;
        ...
    }
\end{minted}

Evaluating the expression \texttt{<condition>} should return a Boolean value:
\texttt{true} or \texttt{false}. If the returned value is \texttt{true}, statements
inside the if-statement code block, such as \texttt{<statement \#1>} and
\texttt{<statement \#2>}, will be executed. For the example mentioned at the
beginning of this section, the if-statement code block should be like this:
\begin{minted}{csharp}
    int input = int.Parse (Console.ReadLine());
    if (input % 2 == 0)
    {
        Console.WriteLine ("It is an even number!");
    }
\end{minted}

\begin{minted}{csharp}
    start
    read movie data
    if (movie produced before 1950):
        then print ("It's an old movie!:)
    end
\end{minted}

Many programming languages offer a statement that allows program to choose an
execution path based on a condition. This is the if statement.

% Add a execution path flowchart for an if statment





The `\textless condition\textgreater' expression should always evaluate to a
boolean value: true or false.
\end{document}
