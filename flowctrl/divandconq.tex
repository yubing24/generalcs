\documentclass[../main.tex]{subfiles}
\begin{document}
    \section{Program Construction: Divide and Conquer}
    Having methods in program not only can remove some tedious coding, but also
    improve the reliability and debugability of the program. For small and trivial
    programs, there is usually no strong reason to have methods for them. However,
    as problem gets complicated, there is need for having methods. Tackling such
    problems demands another important skill in programming: divide and conquer.

    The philosophy behinde ``divide and conquer'' is that: all complex and
    complicated problems are made of a lot of small but repetitive problems. If
    these small problems can be solved, it won't be too hard to solve the entire
    problems. As an analogy, cooking a sophiscated dish can be very complex.
    However, if you divide the ``cooking dish'' into many small processes, such
    as preparing ingredients, making spice, and heat control, then this ``cooking''
    process is instantly simpler and approachable.

    Now, let's use divide and conquer to solve a problem.

\end{document}
