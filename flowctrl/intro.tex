\documentclass[../main.tex]{subfiles}
\begin{document}
\section{Introduction}
So far, you have learned how to write programs that read some inputs from user
and print some information out. However, it is rather primitive if we want to
make a bit more complex program. For example, if we want to write a program that
can read a number from user input and tell if that number is even or odd, we
won't be able to do that yet. Another example would be doing something repetitive,
such as printing out all odd numbers that are between 1 and 1000. It would be very
tedious for us to enumerate all odd numbers in this range and print them out 
individually. Thus we need some mechanism that can allow us to handle such job.

Flow control is one of the earliest programming mechanism introduced. In a
procedure-oriented programming paradigm, having flow control adds more flexibility
and complexity to programs so that programs could handle complex jobs as well. The
most common flow control techniques that are used in today's programming are
conditional statements and loops. 

This chapter will start off with introduction to conditional statements, including
\texttt{if}, \texttt{else if}, and \texttt{else} statements. Three loops will be 
introduced as well: \texttt{for}, \texttt{while}, and \texttt{do while} loop.
At the end of this chapter, two loop-control statements will be introduced:
\texttt{break} and \texttt{continue}. These statements are commonly used in all
C-family programming languages. Besides of them, \csharp also has her own special
\texttt{foreach} loop control, which will be left for instructor's decision.

\end{document}
