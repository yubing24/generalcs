\documentclass[../main.tex]{subfiles}
\begin{document}
    \section{While Loop}
Sometimes we do not exactly know when to jump out of a loop, but we know that we
should jump out a loop when certain condition(s) is satisfied.  In this case,
using a for-loop will be a little bit clumpsy and infleximble. For example, Bob
and Jim want to play basketball after school, but Jim wants to finish his
homework before that. Therefore, for Bob, the waiting process will be like this:

while (Jim is doing home work)
    wait for 5 more minutes

This will make Bob keep waiting for Jim until Jim finishes his homework, and Bob
does not need to know how long it will take Jim to finish.

On contrast, if we use a for-loop, things will be a little bit different:

start = 4'o clock
end = 5'o clock
for (time between start and end):
    wait

This will make Bob waiting for one hour for sure, but what if Jim cannot finish
his homework in an hour, or what if Jim finishes his homework in 30 minutes? In
the former case, should Bob force Jim to play basketball with him regardless of
the unfinished homework? In the later case, should Bob keep wait until that hour

Usually, there are two kinds of while loop in C-family programming languages:
1. while loop
2. do-while loop

They work in a very similar way, except they check conditions at different checkpoints.
\end{document}
