\documentclass[../main.tex]{subfiles}
\begin{document}
    \section{Program Construction: Reverse Engineering}
    It is not unusual that newcomer programmers sometimes do not know how to
    start writing a particular program. This section will show one of the basic
    way of solving a programming program: reverse engeineering.

    Example: You are required to write a program that reads two integers from
    user, calculate the quotient and remainders of them, and print it out in a
    certain format. The required programming langauge is C$\sharp$.

    Let's take a look at these requirements and imagine what it should look like
    if we have had this program (bold letters are user inputs):

    \begin{minted}{csharp}
        Enter numerator: <b>25</b>
        Enter denominator: <b>4</b>
        25 / 4 = 6 with 1
    \end{minted}

As you can tell, this program should be able to print out some prompts, read
two integers, and print out the results of calculation.

Let's see how to use reverse engineering to solve this problem:

Firstly, this program prints out some prompts, which means `Console.Write()`
must have been used in this program.

Secondly, this program reads some inputs and interprets them as integers, thus
\texttt{Console.Read()} and \texttt{int.Parse()} must have been used.

Thirdly, if we run this program multiple times, the results of calculation is
in the same format all the time, thus formated string must have been used.

After the analysis above, we are sure that following C$\sharp$ methods should be
used in this program:
\begin{itemize}
    \item \texttt{Console.Write()} and \texttt{Console.WriteLine()}
    \item \texttt{Console.ReadLine()} and \texttt{int.Parse()}
    \item \texttt{string.Format()}
\end{itemize}

Besides of these, since \texttt{Console.Read()} will return the input from user, and
quotient calculation needs integer variables, thus there must be some string and
integer variables storing these values.

Therefore, we need variables like these:
\begin{itemize}
    \item \texttt{string input}
    \item \texttt{int numerator}
    \item \texttt{int denominator}
    \item \texttt{int quotient}
    \item \texttt{int remainder}
    \item \texttt{string output}
\end{itemize}

With these specified, we can start writing our programs:

Step 1: Write down code skeleton:

\begin{minted}{csharp}
    public class MyProgram
    {
        public static void Main (string [] args)
        {
        }
    }
\end{minted}

Step 2: Declare and initialize variables:

\begin{minted}{csharp}
    public class MyProgram
    {
        public static void Main (string [] args)
        {
            string input = "";
            string output = "";
            int numerator = 0;
            int denominator = 0;
            int quotient = 0;
            int remainder = 0;
        }
    }
\end{minted}

Step 3: Write code that prints the output information (prompts and format strings):

\begin{minted}{csharp}
    public class MyProgram
    {
        public static void Main (string [] args)
        {
            string input = "";
            string output = "";
            int numerator = 0;
            int denominator = 0;
            int quotient = 0;
            int remainder = 0;

            Console.Write ("Enter numerator: ");

            Console.Write ("Enter denominator: ");

            Console.Write (output);
        }
    }
\end{minted}

Step 4: Write code that reads input from users and convert input string into integers:

\begin{minted}{csharp}
    public class MyProgram
    {
        public static void Main (string [] args)
        {
            string input = "";
            string output = "";
            int numerator = 0;
            int denominator = 0;
            int quotient = 0;
            int remainder = 0;

            Console.Write ("Enter numerator: ");
            input = Console.ReadLine ();
            numerator = int.Parse(input)

            Console.Write ("Enter denominator: ");
            input = Console.ReadLine ();
            denominator = int.Parse(input);

            Console.Write (output);
        }
    }
\end{minted}

Step 5: Perform the calculation and assign the results to proper variables:

\begin{minted}{csharp}
    public class MyProgram
    {
        public static void Main (string [] args)
        {
            string input = "";
            string output = "";
            int numerator = 0;
            int denominator = 0;
            int quotient = 0;
            int remainder = 0;

            Console.Write ("Enter numerator: ");
            input = Console.ReadLine ();
            numerator = int.Parse(input)

            Console.Write ("Enter denominator: ");
            input = Console.ReadLine ();
            denominator = int.Parse(input);

            quotient = numerator / denominator;
            remainder = numerator % denominator;

            Console.Write (output);
        }
    }
\end{minted}

Step 6: Produce the format string before print it out:

\begin{minted}{csharp}
public class MyProgram
{
    public static void Main (string [] args)
    {
        string input = "";
        string output = "";
        int numerator = 0;
        int denominator = 0;
        int quotient = 0;
        int remainder = 0;

        Console.Write ("Enter numerator: ");
        input = Console.ReadLine ();
        numerator = int.Parse(input)

        Console.Write ("Enter denominator: ");
        input = Console.ReadLine ();
        denominator = int.Parse(input);

        quotient = numerator / denominator;
        remainder = numerator % denominator;

        output = string.Format ("{0} / {1} = {2} with {3}",
                                numerator,
                                denominator,
                                quotient,
                                remainder);

        Console.Write (output);
    }
}
\end{minted}

Now your program is finished and ready to be compiled and run!

\end{document}
