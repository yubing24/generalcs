\documentclass[../main.tex]{subfiles}
\begin{document}
    \section{Chapter Introduction}

    What is computer? What is computer science? What is it about? Generally, people would agree with
    this definition: computer science is the study of computation and its
    applications. However, like many other subjects, computer science also covers a
    wide range of topics. Typically, computer science studies problems like these:

    \begin{itemize}
        \item How to build a faster computer? (Computer architecture)
        \item How to design a robust operating system? (Operating systems)
        \item How to sort a set of poker cards quickly? (Algorithm)
        \item How to tell what user is searching for? (Artificial intelligence)
        \item How to find the shortest path between two cities? (Graph Theory)
    \end{itemize}

    As you can tell, computer science is studying almost every corner of our daily
    life. When you make a video call to your friend on your smartphone, it involves
    the study of network, operating system, graphics, and computer architecture. It
    is fair to say that without computer science, we will not have our digital life
    today.

    The definition of computer can be a little bit tricky. Technically, a computer
    is a machine that can perform computation, such as a digital calculator, cell
    phones, personal computer, cloud storage server, etc. They do not have to be
    electrical either: some ancient mechanical calculators are also considered as
    computer. In a technical definition, a computer is a machine that can perform
    computation.

    Theoretically, a computer is an abstract machine: they do not have to be
    tangible at all. A theoretical computer is the prototype of those technical
    computers. Compared with a technical computer, a theoretical one is more 

    Topic: Computer science and Programming
    People tend to consider computer science and programming as the same thing. In
    fact, this is not the case. Computer science studies the process of computation
    and the application of computation, while programming is mostly about
    implementing that computation and its applications.

    For example, the design of an operating system is usually the job of a computer
    scientist while the implementation of that design is the work of a programmer.
    A computer scientist, like Bill Gates or Linus Torvalds, comes up with an idea
    about how the system should be designed and run, and a group of programmers will
    write the corresponding code that matches the requirements of the design. If we
    call the design of an operating system as the ``theoretical computer science'',
    then the implementation of that design is the ``experimental computer science''.

    Programming is really a small subset of computer science. Computer scientists
    generally have some programming skills, while programmers' understanding of
    computer science varies a lot.


\end{document}
