\documentclass[../main.tex]{subfiles}
\begin{document}
\section{Basic Computer Architecture}
In order to talk with computers, we need to use programming langauges.

\subsection{Hardware}
The hardware of a computer are usually the tangible parts of it, such as its monitor,
keyboard, mouse, speaker, camera, processor, memory chips, etc. The essential part
of modern computer consists of five basic components: processor, memory, input,
output, and storage. The study of hardware components mostly fall into the study
of electronics and electornic engineering. In these field, people study how to
build effective and efficient computers using those fundamental physics laws.

\subsection{Software}
The intangible part of a computer, such as the operating system, the word processor,
the music player, and web browser are software. They are programs that uses
the resources given by hardware to perform computing works.

\subsection{Operating System}
Operating system is a special software that is responsible for allocating hardware
resources among a variety of software: how much memory can an application use,
which file that an application can modify, which application to run in the next
10 ms, etc. All these detailed work are done by the operating system instead
of other application.

Today, the most three common operating systems that are heavily used are Microsoft
Windows, Mac OS X, and GNU/Linux.

\subsection{User Programs}
User programs are software that mostly for user's special needs, such as word
processor, multimedia player, graphical design, etc. These applications must be
run on top of a specific operating system.

\end{document}
