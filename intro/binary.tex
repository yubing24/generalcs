\documentclass[../main.tex]{subfiles}
\begin{document}
    \section{Binary Representation}
    Type System
Suppose you have 4 light bulbs are connected in this fashion. As you can imagine,
you can turn all these light bulbs on or off, or turn on some of them while leave
the others off. Each light bulb only has two states: on and off. We represent the
``on'' state as 1 and ``off'' state as 0. Therefore, for 4 light bulbs, the
combination of their states make a group of new states. Let's label them with
alphabetic letters like this:
\begin{itemize}
    \item 0000 `A'
    \item 0001 `B'
    \item 0010 `C'
    \item 0011 `D'
    \item 0100 `E'
    \item 0101 `F'
    \item 0110 `G'
    \item 0111 `H'
    \item 1000 `I'
    \item 1001 `J'
    \item 1010 `K'
    \item 1011 `L'
    \item 1100 `M'
    \item 1101 `N'
    \item 1110 `O'
    \item 1111 `P'
\end{itemize}

As you can tell, with only 4 light bulbs, you can make a combination of 16
states, which is \(2^4\) states. If you don't like using them to represent letters,
you can use them to represent numbers, such as:

\begin{itemize}
    \item 0000 0
    \item 0001 1
    \item 0010 2
    \item 0011 3
    \item 0100 4
    \item 0101 5
    \item 0110 6
    \item 0111 7
    \item 1000 8
    \item 1001 9
    \item 1010 A, which is 10
    \item 1011 B, which is 11
    \item 1100 C, which is 12
    \item 1101 D, which is 13
    \item 1110 E, which is 14
    \item 1111 F, which is 15
\end{itemize}

Does it look familiar to you? If not, this is the hexadecimal representation of
numbers.

You can label these states to make them mean something. For example, one way to
label them as alphabetic letters. In this case, we can label 16 English letters
from A to P. If you have more light bultbs, say 5, which allows you to get \(2^5 = 32\)
states. This is definitely enough for you to represent the entire alphabetic
letters just by using on and off of those light bulbs. Yes, it will be a little
bit tedious for you to translate the states of light bulbs into letters, but we
can make a machine that do this translation for us.

Imagine that if you have more lightbulbs, following the induction above, you can
label \(2^64\) different states, which is more than \(1.8 * 10^19\)! This is a huge
number! With 64 light bulbs, you can not only use them to represent alphabetical
letters, but also many characters and letters in other languages, even a large
range of numbers!

The more light bulbs you have, the more information you can represent with these
light bulbs. Each light bulb has only two states. In computer science, we call
this ``binary'' representation of information, and each light bulb is a ``bit''.


\subsection{Primitive Type}
With so many values that a computer can represent, it is better to categorize
them so that manipulate them can be easier and faster. There are many way that
you can categorize them, but one way in particular that computer scientists use
is called type system. Type system is basically categorizing all kinds of values
into following categories:

\begin{itemize}
    \item Integer
    \item Decimal-point number
    \item Character
    \item Logical values
\end{itemize}

Types in the list above are generally refered as \emph{primitive types}. They
are ``primitive'' not because they are primitive, but because their variables
directly contains them.


\subsection{Reference Type}
Besides of primitive type values, computer can also represent another type
called \emph{reference type}. Reference is not a particular kind of type but a
general name of a variety of types. They are called ``reference type'' because
their memory location does no contain values it has, but a memory address that
contains those values. In another word, when you open the box that stores a
reference type value, you will get a little piece of paper telling you which
box that the actual values are in. This may sounds a little bit weird to you at
this stage, but you will understand why reference type is very important in
programming when you understand some basic computer architecture and operating
systems.

A typical reference type in programming is string. A string is a sequence of
characters. For example, ``This is a great day!'' is a string that contains 20
characters. As you may guess, a string can contain arbitrary number of
[characters][1]. Since the size of a string can change all the time, it is
better to store it somewhere else instead of within a single box of memory space.

[1]: Strictly speaking, the number of characters in a string is limited by the
size of computer's avaiable memory.
\end{document}
