\documentclass[../main.tex]{subfiles}
\begin{document}
    \section{Introduction}
    Computer science is not simply about writing a program that can solve a
    particular program. In fact, quite a large portion of computer science is
    about optimization. How to do things more efficiently? In which situation do
    we need to prefer one solution over another? How to measure the efficiency
    of a particular solution? To answer these question, we need to study some
    basic algorithm. Algorithm is all about how to solve some problem
    efficiently or averagely efficiently. This chapter will introduce some basic
    algorithms for two typical problems: sorting and searching.

    Sorting is a kind of problem that we want to rearrage disordered items to
    put them in incremental order. There are many great sorting algorithms that
    have been developed so far, including quicksort, merge sort, and heap sort.
    In this chapter, we will study two kinds of sort: insertion sort and merge
    sort.

    Searching is a kind of problem that we want to find the first occurence of
    a pattern in a larger environment. Typically, both pattern and environment
    are strings. There are two kinds of searching algorithms that are commonly
    used today: Boyer-Moore's algorithm and Kruth's algorithm. However, both of
    them are rather complex for an introductory level text. Therefore, this
    chatper will only introduce the naive algorithm and we will see why it is
    inefficient in most cases.

\end{document}
