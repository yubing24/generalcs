\documentclass[../main.tex]{subfiles}
\begin{document}
    \chapter{Set up Programming Environment}
    Programming is not arcane at all if you have set up the right tools for
    this work. To programming in \csharp, all you need is just two programs: a
    text editor and a compiler.

    \section{Text Editor}
    A text editor is simply a program that can edit text files. Unlike other
    documents such as a Microsoft Word document or a LibreOffice Writer document,
    a text file does not contain any sophisticated format such as page margin
    or font style. Instead it usually only has one font style and very basic
    format. Since programs must be written in plain text in order for compiler
    to recognize and compile them, it is very important for us to write programs
    in text editor.

    There are a variety of text editors avaiable for many platforms, including
    Windows, OS X, and Linux. This book mainly uses an open-source text editor:
    atom. You can download it and install it from this link: http://atom.io.

    Once you have a text editor installed, you can start writing programs. However,
    unless you have a compiler or implement your own \csharp compiler, you still
    cannot turn your source code into an executable file for your computer to run.

    \section{\csharp Compiler}
    Depending on the operating system that you use, you have different options of
    \csharp compiler.

    \emph{Mono} is a free implementation of the \csharp programming language. Its
    compiler is available for download and use for free. You can visit
    http://www.mono-project.com/download/ for more detailed installation guide.
\end{document}
