\documentclass[../main.tex]{subfiles}
\section{Exercise}
\begin{enumerate}
    \item Given a function \(f(x) = x^2 + 22\), what type of value do you think
    is suitable for \(x\)? Is 5 a good value for \(x\)? Is 9.4 a good value for
    \(x\)? Is 1.41421... (the square root of 2) a good value?
    \item Given a function \(f(x) = x\), what type of value do you think is
    suitable for \(x\)? Is the word ``popular'' a good value to plug in to this
    function? Why or why not? Do you think the type of variable depends on what
    the function is doing in this question? Why or why not?
    \item List all the differences between an integer-typed variable and a
    double-typed variable, and how they are represented in a computer.
    \item List all the differences between an character-typed variable and an
    integer-typed variable, and how they are represented in a computer.
    \item Which of the following values are suitable for an integer-typed
    variable? ``word'', `T', 99.38, -121, ``5'', `5'.
    \item What is type-casting? When do you need it?
    \item Explain how to cast an integer value to a character value, using ASCII
    table.
    \item Explain how to cast a character value to an integer value, using ASCII
    table.
    \item Explain what will happen when casting 16.77 to an integer value.
    \item Explain what will happen when casting 99 to a floating-point value.
    \item What is a variable? How many type(s) can a variable have in a program?
    \item Given a line code below: double potatoPrice = 6.44; Can you tell what
        the type, name, and value of this variable is?
    \item What is a declaration of variable? How to declare a variable?
        Assume there is a type called \"Fruit\", declare a variable called
        \"myFruit\" of type \"Fruit\".
    \item What is the assignment of variable? How to assign a value to a variable?
    \item Assume that there is a type called \"double\". Write a line of code
        that declares a variable whose name is \"temperature\" and whose type
        is \"double\", and assigns 35.44 to this variable.
    \item Assume that you have three integer variables: \(price\), \(quantity\),
    and \(cost\), write some code that declares these three variable, and assign
    20 to \(price\), 5 to \(quantity\), and the product of \(price\) and \(quantity\)
    to \(cost\). Try to declare and assign them in as many different order as
    possible, what do you find? Does your code work in all these situations? Why
    or why not? Does the sequence of declaration ans assignment matter?
\end{enumerate}
