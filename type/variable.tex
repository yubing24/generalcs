\documentclass[../main.tex]{subfiles}

\section{Variables}
\begin{document}
When a program is running, there is a lot of data manipulation going on in the
processor and memory of it. While some data are constant that remain the same
value while programming is running, others are changable, and they are usually
refered as `variable'. In computer, a variable is a memory location that is
associated with a symbolic name.

An analogy in math is function. For a linear function such as:\[f(x) = x + 1\]

In the expression above, \(x\) is considered as a variable: you can assign any
numerical value to it, and that expression will just evaluate what ever you assign
to \(x\). Of course, you can also assign the result of the function above to
another variable, say \(y\), and the expression will look like this:
\[y = x + 1\]

If we have a variable \(x\) and the valued stored in it is 20, then we will get
21 if we plug \(x\) in the expression above. If we have another variable \(y\), we
can assign 21 to it and \(y\) will have value 21.

However, you should also notice that you can only plug in a numerical value to
\(x\). It does not make any sense to evaluate the sum of ``banana'' and 1. This
is a very important lesson for us: sometimes a function can only accept certain
kind of input, and that kind is usually called ``type'' in computer science. In
this case, our function above can only accept numerical inputs. In another word,
\(x\) must be a number.

In programming, variables can be created by programmer or program, depends on
the job that the programm wants to perform. Variables must be \emph{declared}
before they can be \emph{referred} or \emph{assigned}. The \emph{declaration} of
a variable is the creation of that variable: programmer ask the operating system
to allocate a chunk of memory to store some value. The \emph{reference} of a
variable is to obtain or use the value stored in that variable. The \emph{assignment}
of variable is to assign a new value to a particular variable. Remember, you must
declare the variable first before you can either refer to it or assign value
to it.

% Now introduce the variable in C#.
\subsection{Variable Declaration in C$\sharp$}
C$\sharp$ is a statically-typed programming langauge, meaning that every variable
you declared in C$\sharp$ programming language must have a type.

\subsection{Variable Naming}
When your program is running, the program accesses variables that you allocated
by using their memory addresses instead of those symbolic names that you put in
code. You may wonder: if the program does not necessarily need the name of
variable, why do we care about the naming of it?

The truth is, your program is not only just for solving a programming problem.
If you are working with a large programming team, your code is very likely to
be read by tens, if not hunderds or thousands, of people, and people need to
understand what your code is doing even when you left the team. Having good
naming of variables make your code self-documented, thus brings good code and
maintainable program.

Compare the following pairs of naming:
\begin{itemize}
    \item An integer variable for age: age and x
    \item A string variable for user input: input and s
    \item A double variable for balance of savings account: savingBalance and b1
\end{itemize}
By comparing the naming choices above, you probably have noticed that good
naming choice for variables can be very helpful in programming. 

\end{document}
