b\documentclass[../main.tex]{subfiles}
\begin{document}
    \section{Real Number}
    In computer, real numbers are numbers that are not integers. This definition
    does not really match the definition of real number in mathematics. Another
    important difference between the definition of real number in computer science
    and in mathematics is that: real numbers in computer are discrete, while real
    numbers in mathematics are continuous. This is mostly because computer only
    has a limited precision, and most real numbers, such as \(1/3\) or $\pi$, has
    infinite number of digits after the decimal point. In computer, only the first
    few digits can be represented. For example, in Python, a very popular
    programming language, the default value of $\pi$ is 3.141592653589793.

    Depending on the precision, there are typically two types of real number in
    computer: \texttt{float} and \texttt{double}. For the \csharp programming
    language, there is another type for real number: \texttt{decimal}.
\end{document}
