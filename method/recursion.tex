\documentclass[../main.tex]{subfiles}
\begin{document}
    \section{Recursion}
    Upcoming
    \subsection{Stack Overflow}
    You probably have heard of this. There is a famous programming Q\& A website
    called StackOverflow, but what actually it is?

    Let's take a look at the following code. Do you think it can be compiled?
    If yes, what do you think it will happen when you run it? Try write down
    this code and run it on your computer:

    \begin{minted}{csharp}
    using System;

    public class Program
    {
        public static void Main (string [] args)
        {
            MethodA();
        }

        public static void MethodA ()
        {
            Console.WriteLine ("Method A is being called!");
            MethodB();
        }

        public static void MethodB ()
        {
            Console.WriteLine ("Method B is being called!");
            MethodA();
        }
    }
    \end{minted}

    Yes, it will be compiled. The \csharp compiler will not complain anything.
    However, when you run it, something weird happens: it seems like the program
    is just printing \texttt{Method A is being called!} and
    \texttt{Method B is being called!} infintely, but if you wait for a while,
    it start printing something like \texttt{at Program.MethodA () <0x0001b>}
    and \texttt{at Program.MethodB () <0x0001b>} infintely. Finally, if you wait
    long enough, the last two lines printed out are something like this:
    \texttt{at Program.Main (string[]) <0x0000b>} and
    \texttt{at (wrapper runtime-invoke) <Module>.runtime\_invoke\_void\_object (object,intptr,intptr,intptr) <0xffffffff>}.

    We won't explain what they mean here: you will learn their meanings when you
    take a course in computer architecture and operating system. But we will
    explain what it tells you.

    When your program is running, there is a chunk of memory, called \emph{stack},
    that is keep tracking of the method calls in your program. Everytime a method
    is called, some data is added to the stack. In our example above, when program
    starts, the \texttt{Main()} method is called, and some data is added to the
    stack. Then the program execute method \texttt{MethodA()}, thus some more
    data is added to stack. Since \texttt{MethodB()} is called in \texttt{MethodA()},
    more data is added to stack. Then you will notice: \texttt{MethodB()} will call
    \texttt{MethodA()}, and in turn \texttt{MethodA()} will call \texttt{MethodB()}.
    This will just keep adding a lot of data to the stack.

    However, the size of the stack is limited: it can only contain certain amount
    of data. When there is too much data on the stack, the stack will overflow.
    When stack is overflown, it will remove all the data in it, print them out,
    and terminat the program. This is called \emph{stack overflow}. In this case,
    stack overflow is caused by recursion: recusively calling \texttt{MethodA()}
    and \texttt{MethodB()}.

\end{document}
