\documentclass[../main.tex]{subfiles}
\begin{document}
    \section {Introduction}
    Modern program consists of thousands even millions line of code. The Linux
    operating system kernel consists of more than 10 millions line of code, and
    most of the code are regular routines: things that needs to be done repetitively.
    \emph{Method (or function in other programming langauges)} supplies one of
    the most important ways that can reduce the complexity of code. When a series
    of tasks need to be performed repetitively, it is better to write a method
    that can handle this job, and everytime the program just need to invoke a
    method when the routine work is needed.

    The most common method that we have used so far in \csharp is the
    \texttt{Main ()} method. The \texttt{Main ()} method is where a program starts
    its execution. Let's take a look at the characteristics of the \texttt{Main ()}
    method.

    \subsection{The Main Method}
    All of our programs so far are written like this:
    \begin{minted}{csharp}
    using System;
    public class Program
    {
        public static void Main (string [] args)
        {
            ...
        }
    }
    \end{minted}

    As you can tell, there are quite a few things associated with the \texttt{Main()}
    method. They are:
    \begin{enumerate}
        \item It is \texttt{public}
        \item It is \texttt{static}
        \item It has a \emph{type} \texttt{void}
        \item It has an \emph{argument}, which is an array of \texttt{string}
    \end{enumerate}

    Therefore, we can infer that a method in \csharp should have the keywords
    \texttt{public} and \texttt{static}. A method should have a \emph{type} as
    well. Optionally, a method should have some \emph{argument{s}}, depending on
    what the method needs to do.


\end{document}
