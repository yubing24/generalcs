\documentclass[../main.tex]{subfiles}
\begin{document}
\section{Programming Languages}
Though we have designed and implemented thousands of programming languages,
there is no such thing called ``perfect programming language'' yet. This is
because all programming languages are designed to achieve a certain goal or
address a solution to a particular issue. Therefore, all programming languages
have its strength and weakness. If you are going to major in computer science or
be a professional programmer, it is very likely that you will master 5 or 6
programming languages in order to perform your work. By that time, you will
learn what language is suitable for performing what kind of tasks.

This book uses a rather popular programming language: C\#. C\# is designed and
mainly implemented by Microsoft Corp. In a long time of its life, it is
considered as a improved version of Java, which is a programming language that
appeared before C$\sharp$.

Extra Topic: The History of Java
Java is a programming language designed and implemented by James Gosling and his
colleagues when they were working for Sun Microsystem Co. in 1995. Unlike many
other programming languages at that time, Java was designed to run on multiple
platforms. Today, Java applications almost run on every device you use: your
laptop PC, your Android smartphone, the Web, etc..

How to Write a C$\sharp$ Program
Everything must follow a set of rule in order to work correctly, especially
computer programs. In fact, programs are so sensitive that even a tiny little
errors can cause huge problems. When you write your programs from this book,
you must copy exactly the words, character to your texteditor and


\subsection{Your First C$\sharp$ Program}
Let's get started by writing a simple C$\sharp$ program: print out some text on
your screen. Suppose we want to print out ``Hello, world!'' on your screen, and
the corresponding program will be like this:

\begin{minted}{csharp}
using System;
public class Program
{
    public static void Main (string [] args)
    {
        Console.WriteLine("Hello, world!");
    }
}
\end{minted}


\end{document}
