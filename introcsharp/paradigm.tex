\documentclass[../main.tex]{subfiles}
\begin{document}
    \section{Programming Paradigm}
    In real life, problems generally have more than one solutions, so do
    programming. In programming, these solutions are categorized into different
    paradigms. Each paradigm is a perspective of solving a problem. Today, there
    are three major programming paradigms: imperative programming, object-oriented
    programming, and functional programming. This book mostly focus on the first
    two programming paradigms.

    Imperative programming is one of the oldest programming paradigm. It views
    solving a problem as giving instructions to computer. In another word,
    solving a problem is executing a group of procedures, thus this programming
    is sometimes called procedural programming. Typical programming languages that
    supports this programming paradigms are: C, FORTRAN, Go, and Pascal.

    Object-oriented programming uses a different view. It considers a problem is
    made of many \emph{objects}, and there is interaction among them. Thus a
    programming problem turns into how to make these objects interact with each
    other. In this programming paradigm, everything, especially data, is viewed
    as some sort of \emph{object}.

    \csharp is a programming language that can supports many different programming
    paradigms: imperative, object-oriented, and a little bit functional programming.
\end{document}
