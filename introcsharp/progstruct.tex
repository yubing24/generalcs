\documentclass[../main.tex]{subfiles}
\begin{document}
    \section{Program Structure: Basics}
    Every C$\sharp$ program must follow a certain structure. In this book, most
    of the programs that you write are executable, meaning that after compiling
    them to lower level code they can be run by your computer directly. For all
    these executable programs, they are all in this form:

\begin{minted}{csharp}
    public class <class identifier>
    {
        public static void Main (string [] args)
        {
        }
    }
\end{minted}

If we have a program whose name is ``Shopping'', you should replace
\texttt{<class identifier>} by the program name. This program should be written
in this form:

\begin{minted}{csharp}
    public class Shopping
    {
        public static void Main (string [] args)
        {
        }
    }
\end{minted}

A few things must be memorized:
\begin{enumerate}
    \item All program should start from \texttt{public class}, followed by a
    pair of curly braces.
    \item All executable programs have a \emph{method}called
    \texttt{Main (string [] args)}. This method should also be
    \texttt{public static void Main (string [] args)}.
\end{enumerate}


When you write your program on a text editor, you should follow these steps:

Step 1. Write down libraries and the class and its braces:

\begin{minted}{csharp}
    using System;
    public class MyProgram
    {
    }
\end{minted}

Step 2. Write the \texttt{Main (string [] args)} between the curly braces. Also,
notice that \texttt{Main (string [] args)} has its own pair of curly braces,
nested in the curly braces of the class.

\begin{minted}{csharp}
    using System;
    public class MyProgram
    {
        public static void Main (string [] args)
        {
        }
    }
\end{minted}
This is the code skeleton that you are going to use for the first few chapters in this book.

\end{document}
