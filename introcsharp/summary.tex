\documentclass[../main.tex]{subfiles}
\begin{document}
    \section{Summary}
    We introduce the basic I/O mechanism in the \csharp programming language.
    To print out message in terminal, use methods:
    \begin{itemize}
        \item \texttt{Console.WriteLine (string value)}
        \item \texttt{Console.Write (string value)}
    \end{itemize}

    \par To read input from user, use methods:
    \begin{itemize}
        \item \texttt{Console.ReadLine ()}
        \item \texttt{Console.Read ()}
    \end{itemize}
    Reading inputs means that new values are added into program while program is
    running, thus it is important to remember saving the input to a \texttt{string}
    variable:
    \begin{minted}{csharp}
        string inputmsg = Console.ReadLine ();
    \end{minted}
\end{document}
