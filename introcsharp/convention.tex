\documentclass[../main.tex]{subfiles}
\begin{document}
\section{Programming Convention}
Like many other discipline such as mathematics, physics, and chemistry, computer
science has her standards of notation and programming conventions. Keeping good
and consistent programming style makes your code easier for your fellows to read
and maintain, just a like a uniform standard notation in chemical elements makes
the communication among chemists easier. Unlike other sciences that only has one
single convention, there many programming conventions and styles for you to
choose. \emph{However, it is very important that you pick one style and
convention and stay with it as long as you can.} This section will introduce
some fundamental conventions in programming and code writing.

\subsection {The 80-Column Rule}
\subsection {Indentation}
Indentation helps you grouping your code into blocks so that you can easily tell
which statements are executed together. There is a debate about the side of
indentation: some organizations, like the Linux kernel development team, prefers
8-character indentation, while some programming langauges, such as Scala, set
the indentation as 2-character long. They both have their rationales. For
C$\sharp$, I recommend to use either 4 or 8-character indentation.

There is one thing that people commonly agree with: you should never have too
many indentations in your code, and the Linux kernel development team even states
that a program shall never have more than three levels of indentation. The reason
behind this is: high level indentation usually comes from nested conditional
statements, such as nested \texttt{if-else}, \texttt{for} loop, and
\texttt{while} loop, and such nested logical control can be greatly simplified
by using some external parameters or functions. 

\end{document}
