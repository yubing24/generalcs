\documentclass[../main.tex]{subfiles}
\begin{document}
    \section{Console I/O}
Programs are written for human use, thus they must be able to communicate with
while running. Such communication involves a mechanism called I/O (stands
for ``input and output''). Programs generally use different form of I/O to
communicate with the outside world. When you use your word processor to write
paper for your history classes, your word processor give you a graphical user
interface (GUI), and what you see on the screen, such as menu and the content
you typed in are the ``output'' of the program. When you type your paper into
the word processor, the word processor will receive inputs from your keyboard.

However, there are other forms of I/O that does not have a graphical user
interface but text-only interface. The only way that you can interact with these
kind of programms in by typing commands and program will perform certain tasks
according to your command. This is called command-line user interface (CLI). In
this book, all programs that you are going to write are in CLI. Most CLI programs
run in console (or terminal). If you are using Microsoft's Windows operating
system, it will be run in a command line environment. If you are using Apple's
Mac OS X or a distribution of Linux, it will be run in your terminal.

Output: Print Text in Command Line
The first thing we want to do is to print some messages on the screen. Let's
make this program print out a sentence: ``No legacy is so rich as honesty'' from
Willam Shakespeare.

Based on our discussion before, every time you write your C$\sharp$ program, you
should start from this template:

\begin{minted}{csharp}
    using System;
    public class MyProgram
    {
        public static void Main (string [] args)
        {
        }
    }
\end{minted}

Let's name this program as \texttt{ch03\_ex01.cs}.

As you can tell, this program is not going to print out the sentence that we
want, since we don't have any statements that tells the computer to print out
things. In C$\sharp$, we use a \emph{method} called \texttt{Console.WriteLine()}
to print out the information that we want.

First of all, let's declare a string variable so that we can store the content
somewhere in this program:

\begin{minted}{csharp}
    using System;
    public class MyProgram
    {
        public static void Main (string [] args)
        {
            string sentence = "No legacy is so rich as honesty.";
        }
    }
\end{minted}

Now we use the function we mentioned: \texttt{Console.WriteLine()} to print out
this message in console/terminal:

\begin{minted}{csharp}
    using System;
    public class MyProgram
    {
        public static void Main (string [] args)
        {
            string sentence = "No legacy is so rich as honesty.";
            Console.WriteLine (sentence);
        }
    }
\end{minted}

\end{document}
