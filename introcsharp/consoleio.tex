\documentclass[../main.tex]{subfiles}
\begin{document}
    \section{Console I/O}
    Programs are written for human use, thus they must be able to communicate with
    the outside environment while running. Such communication involves a mechanism
    called I/O (stands for ``input and output''). Programs generally use different
    form of I/O to communicate with the outside world. When you use your word
    processor to write paper for your history classes, your word processor give
    you a graphical user interface (GUI), and what you see on the screen,
    such as menu and the content you typed in are the ``output'' of the program.
    When you type your paper into the word processor, the word processor will
    receive inputs from your keyboard. I/O is a very important aspect of modern
    computer.

    Besides of the graphical user interface you see on your screen, there are
    other forms of I/O that does not have a graphical user interface but simply
    based on text. The only way that you can interact with these kind of programs
    is typing commands or inputs and program will read what you typed and perform
    certain jobs according to your input. This is called command-line user
    interface (CLI). In this book, all programs that you are going to write are
    in CLI. Most CLI programs run in console (or terminal). If you are using
    Microsoft's Windows operating system, it will be run in a command line
    environment. If you are using Apple's Mac OS X or a distribution of Linux,
    it will be run in your terminal.

    \subsection{Output: Print Text in Terminal}
    The very first thing we want to do is to print some messages on the screen.
    Let's make this program print out a sentence: ``No legacy is so rich as honesty'' from
    Willam Shakespeare.

    First of all, let's start with our template:

    \begin{minted}{csharp}
    using System;
    public class MyProgram
    {
        public static void Main (string [] args)
        {
        }
    }
    \end{minted}

    Let's name this program as \texttt{ioprac01.cs}, which stands for ``I/O
    Practice No.1''.

    Compile this program and run it, and you immediately notice that this
    program does not do anything! This is because we have not tell what computer
    needs to print out yet. In order to tell computer to print out something, we
    need to use a \emph{method}, \texttt{Console.WriteLine(string value)},
    to print out the information that we want.

    First of all, let's declare a string variable so that we can store the content
    somewhere in the memory:

    \begin{minted}{csharp}
    using System;
    public class MyProgram
    {
        public static void Main (string [] args)
        {
            string message = "No legacy is so rich as honesty.";
        }
    }
    \end{minted}

    Now we use the method we mentioned: \texttt{Console.WriteLine()} to print out
    this message in console/terminal:

    \begin{minted}{csharp}
    using System;
    public class MyProgram
    {
        public static void Main (string [] args)
        {
            string message = "No legacy is so rich as honesty.";
            Console.WriteLine (message);
        }
    }
    \end{minted}

    Compile this program and run it, you will see the output like this:

    \texttt{No legacy is so right as honesty.}

    Now you know that everytime you want to print out some message, simply use
    the \texttt{Console.WriteLine(string value)} method!




\end{document}
